\section{About}\label{index_About}
This is TGA's manual main page. TGA is GNU/GPL life and death solver running under Linux operating systems. TGA was programmed in C++ using standard liberies and stl containers. \section{Documentation hiearchy}\label{index_mainpage__sec_1}
TGA's documentation consists of:\begin{itemize}
\item Source code, which is legibly written and thoroughly commented.\item Doxygen generated documentation about main classes, data structures, source files and functions.\item These manual pages\end{itemize}


In these manual pages general relations among program modules are described. Moreover used programming techniques and algorithms are shown together with short source code examples. These code examples are always related to the closest paragraph above them. TGA's documentation is built for purpose of future changes in program ( not neccessarily done by author ). Therefore after reading this manual, then examining classes description and spending little time in particular code reading an average programmer should be able to comfortably operate program sources. Similarly when there is some problem to solve, programmer at first finds adequate topic in manual pages and then follows links to functions and classes managing this topic in the source code ( functions are shortly described in doxygen generated documentation as well ).

Most important manual pages ( and related source files ) are:\begin{itemize}
\item {\bf Handling Parameters}{\rm (p.\,\pageref{page_1})} : {\tt p\_\-manager.cc} {\tt p\_\-manager.h} \item {\bf Starting search}{\rm (p.\,\pageref{page_2})} : {\tt main.cc} {\tt main.h} \item {\bf Tree search}{\rm (p.\,\pageref{page_3})} : {\tt t\_\-search.cc} {\tt t\_\-search.h} \item {\bf Position representation}{\rm (p.\,\pageref{page_6})} : {\tt board.cc} {\tt board.h} \item {\bf Static eye recognition}{\rm (p.\,\pageref{page_4})} : {\tt eyes.cc} {\tt eyes.h} \item {\bf Sgf parsing}{\rm (p.\,\pageref{page_5})} : {\tt sgf\_\-read.cc} {\tt sgf\_\-read.h} \end{itemize}
