\section{Handling Parameters}\label{page_1}
Parameters are handled by class {\tt Params\_\-manager} declared in {\tt p\_\-manager.h} and implemented in {\tt p\_\-manager.cc}. The idea behing is quite simple. At first ( in {\tt Params\_\-manager::Params\_\-manager()} ) a table where each record represents one parameter ( {\tt Param\_\-record} ) is created. This table is filled with particular parameters and their records ( parameter token, parameter alternative token, parameter type, a pointer to the corresponding variable ).\begin{itemize}
\item Parameter token is the name of parameter itself ( e.g. \char`\"{}-i\char`\"{}, \char`\"{}-ntt\char`\"{} )\item Parameter alternative token is the second name of parameter itself, in GNU style ( e.g. \char`\"{}--input\char`\"{}, \char`\"{}--no\_\-t\_\-tables\char`\"{})\item Parameter type tells whether the parameter value is of type {\tt bool} ( here two values for filling adequate variable with {\tt true} and {\tt false} are distinguished - see {\tt Param\_\-type} ) or {\tt string} or {\tt int} \item Pointer to the corresponding variable that serves for quick filling this variable with a new value. The latter algortihm deals only with these variables ( they are defined in {\tt Params\_\-manager} itself )\end{itemize}


After the program is started,the algorithm goes through the given parameters ( {\bf Starting search}{\rm (p.\,\pageref{page_2})} ) and according to {\tt Params\_\-manager::param\_\-table} sets ( done by {\tt Params\_\-manager::set\_\-variable()} ) variables corresponding to particular parameters to their proper values ( usually only sets the variable to {\tt true} : most of parameters are \char`\"{}switchers\char`\"{} ). 