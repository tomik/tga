\section{Starting search}\label{page_2}
All program actions start in {\tt {\bf main()}{\rm (p.\,\pageref{main_8cc_a6})}} function defined in {\tt {\bf main.cc}{\rm (p.\,\pageref{main_8cc})}}. \subsection{After the start}\label{page_2_page_2__sec_1}
At first, program arguments are parsed and variables corresponding to parameters are set to their proper values ( by calling {\tt {\bf Params\_\-manager::set\_\-variable()}{\rm (p.\,\pageref{classParams__manager_a23})}} ). After this, program chooses one of two possible continuations :\begin{itemize}
\item Single position search mode. This is the most commonly used mode. The algorithm simply calls {\tt {\bf handle\_\-single\_\-problem()}{\rm (p.\,\pageref{main_8cc_a3})}} function, which takes care of proper actions in a single problem tree search.\item Multiple positions search mode. The program opens a file with problem names and then goes through them and on each one of them calls function {\tt {\bf handle\_\-single\_\-problem()}{\rm (p.\,\pageref{main_8cc_a3})}}. The difference with single position mode is also in the output: e.g. the information about group status, time of search are differently formated.\end{itemize}
\subsection{Handling a single problem}\label{page_2_page_2__sec_2}
This task is done by function {\tt {\bf handle\_\-single\_\-problem()}{\rm (p.\,\pageref{main_8cc_a3})}} and consists of:\begin{itemize}
\item the initialization of possible persistent transposition tables ( when nor {\tt -ntt} neither {\tt -nst} options are given ) 

\footnotesize\begin{verbatim}if ( params_manager.get_transposition_tables() && params_manager.get_save_tt()) { 
  t_searcher.hash_table_manager.hash_table_activate(); // hash table is initialized
  board_manager.hash_board_manager.init_hash_board();   
}
\end{verbatim}
\normalsize
\item covering neccessary prints\item analysing the single position for the attacker moving first and then for the defender moving first ( if their moves aren't blocked by given argument ). Example of \char`\"{}defender moving first\char`\"{} section: 

\footnotesize\begin{verbatim}if ( params_manager.get_defender_moves_first() ) { // defender moving first 
  ...  // output prints
  status_de = analyze_single_position (  filename , sc_white);

  print_group_status ( sc_white , status_de );
}
\end{verbatim}
\normalsize
\end{itemize}
\subsection{Analysing single position}\label{page_2_page_2__sec_3}
By the term analysing single position is meant to search a problem with a specified color to move first ( attacker or defender ). This task is performed by function {\tt {\bf analyze\_\-single\_\-position()}{\rm (p.\,\pageref{main_8cc_a2})}} and consists of three main steps :\begin{itemize}
\item initialization of data structures: 

\footnotesize\begin{verbatim}sgf_parser.init();
board_manager.init();
... // potential init of transposition tables if these are switched on
\end{verbatim}
\normalsize
\item parsing sgf source of the input position 

\footnotesize\begin{verbatim}sgf_parser.parse_sgf( pos_name.c_str()); // parsing sgf input 
 //board initialization from temporar data structure follows
board_manager.init_from_temp_board_manager( 
  sgf_parser.get_temp_board_manager()); 
\end{verbatim}
\normalsize
\item performing search itself ( for a certain player playing first - see parameters of function {\tt {\bf analyze\_\-single\_\-position()}{\rm (p.\,\pageref{main_8cc_a2})}}) Here must be distinguished whether the option {\tt -id} ( iterative deepening depth first search ) is switched on. If it is, search is performed in the cycle with iteratively increasing maximal depth of the search, until certain status ( the accuracy of the status is {\tt gsa\_\-certain} ) is not obtained. 

\footnotesize\begin{verbatim}if ( params_manager.get_iddfs() ) { // iddfs activated
  while (( result.group_status_accuracy != gsa_certain)){ // iterative search
    ...
    switch ( color_to_move ) {
      case sc_white: 
        sgf_printer.set_output (save_filename);
        result = t_searcher.search(nt_max, 0 , gs_dead, gs_alive, false,false);
        break;
      .. //symetrically for sc_black
    }
    t_searcher.increase_max_depth(); // increase maximal depth of the next search
}
\end{verbatim}
\normalsize
\end{itemize}


If no iddfs is demanded only a single search is performed and afterwards the information on the number of searched nodes together with search time is printed.

All program actions start in {\bf main()}{\rm (p.\,\pageref{main_8cc_a6})} function defined in {\tt {\bf main.cc}{\rm (p.\,\pageref{main_8cc})}}. \subsection{After the start}\label{page_2_page_2__sec_1}
At first, program arguments are parsed and variables corresponding to parameters are set to their proper values ( by calling {\bf Params\_\-manager::set\_\-variable()}{\rm (p.\,\pageref{classParams__manager_a23})} ). After this, program chooses one of two possible continuations :\begin{itemize}
\item Single position search mode. This is the most commonly used mode. The algorithm simply calls {\bf handle\_\-single\_\-problem()}{\rm (p.\,\pageref{main_8cc_a3})} function, which takes care of proper actions in a single problem tree search.\item Multiple positions search mode. The program opens a file with problem names and then goes through them and on each one of them calls function {\bf handle\_\-single\_\-problem()}{\rm (p.\,\pageref{main_8cc_a3})}. The difference with single position mode is also in the output: the information about group status, time of search, ... are differently formated.\end{itemize}
\subsection{Handling a single problem}\label{page_2_page_2__sec_2}
This task is done by function {\bf handle\_\-single\_\-problem()}{\rm (p.\,\pageref{main_8cc_a3})} and consists of:\begin{itemize}
\item the initialization of possible persistent transposition tables ( when {\tt -tt} and {\tt -st} options are given ) 

\footnotesize\begin{verbatim}    if ( params_manager.get_transposition_tables() && params_manager.get_save_tt() ) { // transposition tables and their persistency are switched on 
      t_searcher.hash_table_manager.hash_table_activate();  // hash table and hash keys for board is initialized once for both searches
      board_manager.hash_board_manager.init_hash_board();   
    }
\end{verbatim}
\normalsize
\item covering neccessary prints\item analysing the single position ( that means it is decided who makes the first move ) for the attacker moving first and then for the defender moving first ( if their moves aren't blocked by given argument ). Example of \char`\"{}defender moving first\char`\"{} section: 

\footnotesize\begin{verbatim}  if ( params_manager.get_defender_moves_first() ) { // defender moving first is not blocked 
    ...  // output prints
    status_de = analyze_single_position (  filename , sc_white);
    ...  // output prints
    print_group_status ( sc_white , status_de );
  }
\end{verbatim}
\normalsize
\end{itemize}
\subsection{Analysing single position}\label{page_2_page_2__sec_3}
By the term analysing single position is meant to search a problem with a specified color to move first ( attacker or defender ). This task is performed by function {\bf analyze\_\-single\_\-position()}{\rm (p.\,\pageref{main_8cc_a2})} and consists of three main steps :\begin{itemize}
\item initialization of data structures: 

\footnotesize\begin{verbatim}      sgf_parser.init();
      board_manager.init();
      ... // potential init of transposition tables dependent on set parameters 
\end{verbatim}
\normalsize
\item parsing sgf source of the input position 

\footnotesize\begin{verbatim}    sgf_parser.parse_sgf( pos_name.c_str() ); // parses sgf input and saves it to temp_board_manager
    board_manager.init_from_temp_board_manager( sgf_parser.get_temp_board_manager() ); // correctly inits board_manager from temp_board_maanger
\end{verbatim}
\normalsize
\item performing search itself ( for a certain player playing first - see parameters of function {\bf analyze\_\-single\_\-position()}{\rm (p.\,\pageref{main_8cc_a2})} ) Here must be distinguished whether option {\tt -id} ( iterative deepening depth first search ) is switched on. If it is, search is performed in the cycle with iteratively increasing maximal depth of the search, until certain status ( the accuracy of the status is  gsa\_\-certain ) is not obtained.

\footnotesize\begin{verbatim}    if ( params_manager.get_iddfs() ) { // iddfs activated
      while ( ( result.group_status_accuracy != gsa_certain) ) { // iterative search 
            ...
            switch ( color_to_move ) {
            case sc_white: sgf_printer.set_output (save_filename);
                           result = t_searcher.search(nt_max, 0 , gs_dead, gs_alive, false,false); break;
            ... //symetrically for sc_black
            }
          t_searcher.increase_max_depth(); // increase maximal depth of search
      }
\end{verbatim}
\normalsize
\end{itemize}


If no iddfs is demanded only a single search is performed and afterwards the information on the number of searched nodes together with search time is provided. 